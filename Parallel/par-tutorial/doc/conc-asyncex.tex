\Subsection{async-exceptions}{Cancellation: Asynchronous Exceptions}

In an interactive application, it is often important for one thread to
be able to \emph{interrupt} the execution of another thread when some
particular condition occurs.  Some examples of this kind of behaviour
in practice include:

\begin{itemize}
\item In a web browser, the thread downloading the web page and the
  thread rendering the page need to be interrupted when the user
  presses the ``stop'' button.

\item A server application typically wants to give a client a set
  amount of time to issue a request before closing its connection, so
  as to avoid dormant connections using up resources.

\item An application in which a compute-intensive thread is working
  (say, rendering a visualisation of some data), and the input data
  changes due to some user input.
\end{itemize}

The crucial design decision in supporting cancellation is whether the
intended victim should have to poll for the cancellation condition,
or whether the thread is immediately cancelled in some way.  This is a
tradeoff:

\begin{enumerate}
\item If the thread has to poll, there is a danger that the programmer
  may forget to poll regularly enough, and the thread will become
  unresponsive, perhaps permanently so.  Unresponsive threads lead to
  hangs and deadlocks, which are particularly unpleasant from a
  user's perspective.
\item If cancellation happens asynchronously, critical sections that
  modify state need to be protected from cancellation, otherwise
  cancellation may occur mid-update leaving some data in an
  inconsistent state.
\end{enumerate}

In fact, the choice is really between doing only (1), or doing both
(1) and (2), because if (2) is the default, protecting a critical
section amounts to switching to polling behaviour for the duration of
the critical section.

In most imperative languages it is unthinkable for (2) to be the
default, because so much code is state-modifying.  Haskell has a
distinct advantage in this area, however: most code is purely
functional, so it can be safely aborted or suspended, and later
resumed, without affecting correctness.  Moreover our hand is
forced: purely functional code cannot by definition poll for the
cancellation condition, so it must be cancellable by default.

Therefore, fully-asynchronous cancellation is the only sensible
default in Haskell, and the design problem reduces to deciding how
cancellation appears to code in the @IO@ monad.

It makes sense for cancellation to behave like an exception, since
exceptions are already a fact of life in the @IO@ monad, and the usual
idioms for writing @IO@ monad code include exception handlers to
release resources and clean up in the event of an error.  For example,
to perform an operation that requires a temporary file, we would use
the @bracket@ combinator to ensure that the temporary file is always
removed, even if the operation raises an exception:

\begin{haskell}
  bracket (newTempFile "temp")
          (\file -> removeFile file)
          (\file -> ...)
\end{haskell}

\noindent where @bracket@ is defined thus:

\begin{haskell}
bracket :: IO a -> (a -> IO b) -> (a -> IO c) -> IO c
bracket before after during = do
  a <- before
  c <- during a `onException` after a
  after a
  return c
\end{haskell}

\noindent and @onException@ executes its first argument, and if an
exception is thrown, executes its second argument before re-throwing
the exception.

\begin{haskell}
onException :: IO a -> IO b -> IO a
\end{haskell}

We want exception handlers to run in the event of cancellation, so
cancellation should be an exception.  However, there's a fundamental
difference between the kind of exception thrown by @openFile@ when the
file does not exist, for example, and an exception that may arise
\emph{at any time} because the user pressed the ``stop'' button.  We
call the latter kind an \emph{asynchronous} exception, for obvious
reasons.  (We do not review the Haskell support for \emph{synchronous}
exceptions here; for that see the Haskell 2010 report
\cite{haskell2010} and the documentation for the @Control.Exception@
module).

To initiate an asynchronous exception, Haskell provides the @throwTo@
primitive which throws an exception from one thread to
another \cite{spj:asynch-exceptions}:

\begin{haskell}
throwTo :: Exception e => ThreadId -> e -> IO ()
\end{haskell}

\noindent the @Exception@ constraint requires that the exception value
being thrown is an instance of the @Exception@ class, which implements
a simple hierarchy \cite{extensible-exceptions}.  The @ThreadId@ is a
value previously returned by @forkIO@, and may refer to a thread in
any state: running, blocked, or finished (in the latter case,
@throwTo@ is a no-op).

To illustrate the use of @throwTo@, we now elaborate the earlier
example in which we downloaded several web pages concurrently, to
allow the user to hit @'q'@ at any time to stop the downloads.

First, we will extend our @Async@ mini-API to allow cancellation.  We
add one operation:

\begin{haskell}
cancel :: Async a -> IO ()
\end{haskell}

\noindent which cancels an existing @Async@.  If the operation has
already completed, @cancel@ has no effect.  The @wait@ operation
cannot just return the result of the @Async@ any more, since it may
have been cancelled.  Therefore, we extend @wait@ to return
@Either SomeException a@, containing either the exception raised during the
operation, or its result:

\begin{haskell}
wait :: Async a -> IO (Either SomeException a)
\end{haskell}

\noindent (@SomeException@ is the root of the exception hierarchy in Haskell.)
In order to implement the new interface, we need to extend the @Async@
type to include the @ThreadId@ of the child thread, and the @MVar@
holding the result must now hold @Either SomeException a@.

\begin{haskell}
data Async a = Async ThreadId (MVar (Either SomeException a))
\end{haskell}

\noindent Given this, the implementation of @cancel@ just throws an
exception to the thread:

\begin{haskell}
cancel :: Async a -> IO ()
cancel (Async t var) = throwTo t ThreadKilled
\end{haskell}

\noindent (@ThreadKilled@ is an exception provided by the Haskell exception
library and is typically used for cancelling threads in this way.)
The implementation of @wait@ is trivial.
The remaining piece of the implementation is the @async@ operation,
which must now include an exception handler to catch the exception and
store it in the @MVar@:

\begin{haskell}
async :: IO a -> IO (Async a)
async io = do
   m <- newEmptyMVar
   t <- forkIO (do r <- try io; putMVar m r)
   return (Async t m)
\end{haskell}

\noindent where @try@ is a function provided by the
@Control.Exception@ library:

\begin{haskell}
try :: Exception e => IO a -> IO (Either e a)
\end{haskell}

Now, we can change the @main@ function of the example to support
cancelling the downloads:

\begin{numhaskell}
main = do
  as <- mapM (async.http) sites

  forkIO $ do
     hSetBuffering stdin NoBuffering
     forever $ do
        c <- getChar
        when (c == 'q') $ mapM_ cancel as

  rs <- mapM wait as
  printf "%d/%d finished\n" (length (rights rs)) (length rs)
\end{numhaskell}

\noindent Line 2 starts the downloads as before.  Lines 4--8 fork a
new thread that repeatedly reads characters from the standard input,
and if a @q@ is found, calls @cancel@ on all the @Async@s.  Line 10
waits for all the results (complete or cancelled), and line 11 emits a
summary with a count of how many of the operations completed without
being cancelled.  If we run the sample\footnote{full code is in the
  sample @geturlscancel.hs@} and hit @`q`@ fast enough, we see
something like this:

\begin{verbatim}
downloaded: http://www.google.com (14538 bytes, 0.17s)
downloaded: http://www.bing.com (24740 bytes, 0.22s)
q2/5 finished
\end{verbatim}

Note that this works even though the program is sitting atop a large
and complicated HTTP library that provides no direct support for
either cancellation or asynchronous I/O.  Haskell's support for
cancellation is modular in this respect: most library code needs to do
nothing to support it, although there are some simple and unintrusive
rules that need to be followed when dealing with state, as we shall
see in the next section.

\Subsubsection{mask}{Masking asynchronous exceptions}

As we mentioned earlier, the danger with fully asynchronous exceptions
is that one might fire while we are in the middle of updating some
shared state, leaving the data in an inconsistent state, and with a
high probability leading to mayhem later.

Hence, we certainly need a way to control the delivery of asynchronous
exceptions during critical sections.  But we must tread carefully: it
would be easy to provide the programmer with a way to turn off
asynchronous exception delivery temporarily, but such a facility is in
fact not what we really need.

Consider the following problem: a thread wishes to call @takeMVar@,
perform an operation depending on the value of the @MVar@, and finally
put the result of the operation in the @MVar@.  The code must be
responsive to asynchronous exceptions, but it should be safe: if an
asynchronous exception arrives after the @takeMVar@, but before the
final @putMVar@, the @MVar@ should not be left empty, instead the
original value should be replaced.

If we code up this problem using the facilities we already seen so
far, we might end up with something like this:

\begin{numhaskell}
problem m f = do
  a <- takeMVar m
  r <- f a `catch` \e -> do putMVar m a; throw e
  putMVar m r
\end{numhaskell}

\noindent There are at least two points where, if an asynchronous
exception strikes, the invariant will be violated.  If an exception
strikes between lines 2 and 3, or between lines 3 and 4, the @MVar@
will be left empty.  In fact, there is no way to shuffle around the
exception handlers to ensure the @MVar@ is always left full.  To fix
this problem, Haskell provides the @mask@
combinator\footnote{Historical note: the original presentation of
  asynchronous exceptions used a pair of combinators @block@ and
  @unblock@ here, but @mask@ was introduced in GHC 7.0.1 to replace
  them as it has a more modular behaviour.}:

\begin{haskell}
mask :: ((IO a -> IO a) -> IO b) -> IO b
\end{haskell}

\noindent The type looks a bit confusing, but it isn't
really\footnote{for simplicity here we are using a slightly less
  general version of @mask@ than the real one in the
  @Control.Exception@ library.}.  The @mask@ operation defers the
delivery of asynchronous exceptions for the duration of its argument,
and is used like this:

\begin{numhaskell}
problem m f = mask $ \restore -> do
  a <- takeMVar m
  r <- restore (f a) `catch` \e -> do putMVar m a; throw e
  putMVar m r
\end{numhaskell}

\noindent @mask@ is applied to a \emph{function}, that takes as its
argument a function @restore@, that can be used to restore the
delivery of asynchronous exceptions to its present state.  If we
imagine shading the entire argument to @mask@ except for the
expression @(f a)@, asynchronous exceptions cannot be raised in the
shaded portions.

This solves the problem that we had previously, since now an exception
can only be raised while @(f a)@ is working, and we have an exception
handler to catch any exceptions in that case.  But a new problem has
been introduced: @takeMVar@ might block for a long time, but it is
inside the @mask@ and so the thread will be unresponsive for that
time.  Furthermore there's no good reason to mask exceptions during
@takeMVar@; it would be safe for exceptions to be raised right up
until the point where @takeMVar@ returns.  Hence, this is exactly the
behaviour that Haskell defines for @takeMVar@: we designate a small
number of operations, including @takeMVar@, as \emph{interruptible}.
Interruptible operations may receive asynchronous exceptions even
inside @mask@.

What justifies this choice?  Think of @mask@ as ``switching to polling
mode'' for asynchronous exceptions.  Inside a @mask@, asynchronous
exceptions are no longer asynchronous, but they can still be raised by
certain operations.  In other words, asynchronous exceptions become
\emph{synchronous} inside @mask@.

All operations which may block indefinitely\footnote{except foreign
  calls, for technical reasons} are designated as interruptible.  This
turns out to be the ideal behaviour in many situations, as in
@problem@ above.

In fact, we can provide higher level combinators to insulate
programmers from the need to use @mask@ directly.  For example, the
function @problem@ above is generally useful when working with
@MVar@s, and is provided under the name @modifyMVar_@ in the
@Control.Concurrent.MVar@ library.

\Subsubsection{async-safety}{Asynchronous-exception safety}

All that is necessary for most code to be safe in the presence of
asynchronous exceptions is to use operations like @modifyMVar_@
instead of @takeMVar@ and @putMVar@ directly.  For example, consider
the buffered channels that we defined earlier.  As defined, the
operations are not asynchronous-exception-safe; for example,
@writeChan@ was defined like this:

\begin{numhaskell}
writeChan :: Chan a -> a -> IO ()
writeChan (Chan _ writeVar) val = do
  new_hole <- newEmptyMVar
  old_hole <- takeMVar writeVar
  putMVar writeVar new_hole
  putMVar old_hole (Item val new_hole)
\end{numhaskell}

\noindent there are several windows here where if an asynchronous
exception occurs, an @MVar@ will be left empty, and subsequent users
of the @Chan@ will deadlock.  To make it safe, we use @modifyMVar_@:

\begin{numhaskell}
writeChan (Chan _ writeVar) val = do
  new_hole <- newEmptyMVar
  modifyMVar_ writeVar $ \old_hole -> do
    putMVar old_hole (Item val new_hole)
    return new_hole
\end{numhaskell}

We saw a use of the @bracket@ function earlier; in fact, @bracket@ is
defined with @mask@ in order to make it asynchronous-exception-safe:

\begin{numhaskell}
bracket before after thing =
  mask $ \restore -> do
    a <- before
    r <- restore (thing a) `onException` after a
    _ <- after a
    return r
\end{numhaskell}

\Subsubsection{timeout}{Timeouts}

A good illustration of programming with asynchronous exceptions is to
write a function that can impose a time limit on a given action.  We
want to provide the timeout wrapper as a combinator of the following
type:

\begin{haskell}
timeout :: Integer -> IO a -> IO (Maybe a)
\end{haskell}

\noindent where @timeout @$t$@ @$m$ has the following behaviour:

\begin{enumerate}
\item @timeout @$t~m$ behaves exactly like @fmap Just @$m$ if $m$ returns a
  result or raises an exception (including an asynchronous exception),
  within $t$ microseconds.
\item otherwise, $m$ is sent an asynchronous exception of the form
  @Timeout @$u$.  @Timeout@ is a new datatype that we define, and $u$
  is a unique value of type @Unique@, distinguishing this particular
  instance of @timeout@ from any other.  The call to @timeout@ then
  returns @Nothing@.
\end{enumerate}

The implementation is not expected to implement real-time semantics,
so in practice the timeout will only be approximately $t$ microseconds.
Note that (1) requires that $m$ is executed in the context of the
current thread, since $m$ could call @myThreadId@, for example.  Also,
another thread throwing an exception to the current thread with
@throwTo@ will expect to interrupt $m$.

\begin{lstlisting}[float,label=lst:timeout,caption=implementation of \texttt{timeout},language=HaskellUlisses,style=numbers]
timeout n m
    | n <  0    = fmap Just m
    | n == 0    = return Nothing
    | otherwise = do
        pid <- myThreadId
        u <- newUnique
        let ex = Timeout u
        handleJust
           (\e -> if e == ex then Just () else Nothing)
           (\_ -> return Nothing)
           (bracket (forkIO $ do threadDelay n
                                 throwTo pid ex)
                    (\t -> throwTo t ThreadKilled)
                    (\_ -> fmap Just m))
\end{lstlisting}

The code for @timeout@ is shown in \lstref{timeout}; this
implementation was taken from the library @System.Timeout@ (with some
cosmetic changes for presentation here).  The implementation is tricky
to get right.  The basic idea is to fork a new thread that will wait
for $t$ microseconds and then call @throwTo@ to throw the @Timeout@
exception back to the original thread; that much seems straightforward
enough.  However, we must ensure that this thread cannot throw its
@Timeout@ exception after the call to @timeout@ has returned,
otherwise the @Timeout@ exception will leak out of the call, so
@timeout@ must kill the thread before returning.

Here is how the implementation works, line by line:

\begin{itemize}
\item[1--2] Handle the easy cases, where the timeout is
  negative or zero.
\item[5] find the @ThreadId@ of the current thread
\item[6--7] make a new @Timeout@ exception, by generating a unique value
  with @newUnique@
\item[8-14] @handleJust@ is an exception handler, with the following
  type:
\begin{haskell}
handleJust :: Exception e
           => (e -> Maybe b) -> (b -> IO a) -> IO a
           -> IO a
\end{haskell}
  Its first argument (line 9) selects which exceptions to catch: in
  this case, just the @Timeout@ exception we defined on line 7.  The
  second argument (line 10) is the exception handler, which in this
  case just returns @Nothing@, since timeout occurred.

  Lines 11--14 are the computation to run in the exception handler.
  @bracket@ (\secref{async-exceptions}) is used here in order to fork
  the child thread, and ensure that it is killed before returning.

  \begin{itemize}
    \item[11-12] fork the child thread.  In the child thread we wait
      for $n$ microseconds with @threadDelay@, and then throw the
      @Timeout@ exception to the parent thread with @throwTo@.
    \item [13] always kill the child thread before returning.
    \item [14] the body of @bracket@: run the computation @m@ passed
      in as the second argument to @timeout@, and wrap the result in
      @Just@.
  \end{itemize}
\end{itemize}

The reader is encouraged to verify that the implementation works by
thinking through the two cases: either @m@ completes and returns
@Just x@ at line 14, or, the child thread throws its exception while
@m@ is still working.

There is one tricky case to consider: what happens if \emph{both} the
child thread and the parent thread try to call @throwTo@ at the same
time (lines 12 and 13 respectively)?  Who wins?

The answer depends on the semantics of @throwTo@.  In order for this
implementation of @timeout@ to work properly, it must not be possible
for the call to @bracket@ at line 11 to return while the @Timeout@
exception can still be thrown, otherwise the exception can leak.
Hence, the call to @throwTo@ that kills the child thread at line 13
must be synchronous: once this call returns, the child thread cannot
throw its exception any more.  Indeed, this guarantee is provided by
the semantics of @throwTo@: a call to @throwTo@ only returns after the
exception has been raised in the target thread\footnote{Note: a
  different semantics was originally described in
  \citet{spj:asynch-exceptions}.}.  Hence, @throwTo@ may block if the
child thread is currently masking asynchronous exceptions with @mask@,
and because @throwTo@ may block, it is therefore \emph{interruptible}
and may itself receive asynchronous exceptions.

Returning to our ``who wins'' question above, the answer is ``exactly
one of them'', and that is precisely what we require to ensure the
correct behaviour of @timeout@.

\Subsubsection{async-reflections}{Asynchronous exceptions: reflections}

Abstractions like @timeout@ are certainly difficult to get right, but
fortunately they only have to be written once.  We find that in
practice dealing with asynchronous exceptions is fairly
straightforward, following a few simple rules:

\begin{itemize}
\item Use @bracket@ when acquiring resources that need to be released
  again.
\item Rather than @takeMVar@ and @putMVar@, use @modifyMVar_@ (and
  friends) which have built-in asynchronous exception safety.
\item If state handling starts getting complicated with multiple
  layers of exception handlers, then there are two approaches to
  simplifying things:
  \begin{itemize}
    \item Switching to polling mode with @mask@ can help manage
      complexity.  The GHC I/O library, for example, runs entirely
      inside @mask@.  Note that inside @mask@ it is important to
      remember that asynchronous exceptions can still arise out of
      interruptible operations; the documentation contains a list of
      operations that are guaranteed \emph{not} to be interruptible.
    \item Using Software Transactional Memory (STM) instead of @MVar@s
      or other state representations can sweep away all the complexity
      in one go.  We will describe STM in \secref{stm}.
  \end{itemize}
\end{itemize}

The rules are usually not onerous: remember this only applies to code
in the @IO@ monad, so the vast swathes of purely-functional library
code available for Haskell is all safe by construction.  We find that
most @IO@ monad code is straightforward to make safe, and if things
get complicated falling back to either @mask@ or STM is a satisfactory
solution.

In exchange for following the rules, however, Haskell's approach to
asynchronous exceptions confers many benefits.

\begin{itemize}
\item Many exceptional conditions map naturally onto asynchronous
  exceptions.  For example, stack overflow and user interrupt
  (e.g. control-C at the console) are mapped to asynchronous
  exceptions in Haskell.  Hence, control-C not only aborts the program
  but does so cleanly, running all the exception handlers.  Haskell
  programmers have to do nothing to enable this behaviour.

\item Constructs like @timeout@ always work, even with third-party
  library code.

\item Threads never just die in Haskell, it is guaranteed that a
  thread always gets a chance to clean up and run its exception
  handlers.
\end{itemize}

