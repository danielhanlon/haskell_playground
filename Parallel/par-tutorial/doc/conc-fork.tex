\Subsection{forking}{Forking Threads}

The basic requirement of concurrency is to be able to fork a new
thread of control.  In Concurrent Haskell this is achieved with the
@forkIO@ operation:

\begin{haskell}
forkIO :: IO () -> IO ThreadId
\end{haskell}

\noindent @forkIO@ takes a computation of type @IO ()@ as its
argument; that is, a computation in the @IO@ monad that eventually
delivers a value of type @()@.  The computation passed to @forkIO@ is
executed in a new \emph{thread} that runs concurrently with the other
threads in the system.  If the thread has effects, those effects will
be interleaved in an indeterminate fashion with the effects from other
threads.

To illustrate the interleaving of effects, let's try a simple example
in which two threads are created, once which continually prints the
letter @A@ and the other printing @B@\footnote{this is sample @fork.hs@}:

\begin{numhaskell}
import Control.Concurrent
import Control.Monad
import System.IO

main = do
  hSetBuffering stdout NoBuffering
  forkIO (forever (putChar 'A'))
  forkIO (forever (putChar 'B'))
  threadDelay (10^6)
\end{numhaskell}

\noindent Line 6 puts the output @Handle@ into non-buffered mode, so
that we can see the interleaving more clearly.  Lines 7 and 8 create
the two threads, and line 9 tells the main thread to wait for one
second (@10^6@ microseconds) and then exit.

When run, this program produces output something like this:

\begin{verbatim}
AAAAAAAAABABABABABABABABABABABABABABABABABABABABABABAB
ABABABABABABABABABABABABABABABABABABABABABABABABABABAB
ABABABABABABABABABABABABABABABABABABABABABABABABABABAB
ABABABABABABABABABABABABABABABABABABABABABABABABABABAB
\end{verbatim}

\noindent Note that the interleaving is non-deterministic: sometimes we get
strings of a single letter, but often the output switches regularly
between the two threads.  Why does it switch so regularly, and why
does each thread only get a chance to output a single letter before
switching?  The threads in this example are contending for a single
resource: the @stdout@ Handle, so scheduling is affected by how
contention for this resource is handled.  In the case of GHC a Handle
is protected by a lock implemented as an @MVar@ (described in the next
section).  We shall see shortly how the implementation of @MVar@s
causes the @ABABABA@ behaviour.

% ToDo: move this into a separate section and expand.

We emphasised earlier that concurrency is a program structuring
technique, or an abstraction.  Abstractions are practical when they
are efficient, and this is where GHC's implementation of threads comes
into its own.  Threads are extremely lightweight in GHC: a thread
typically costs less than a hundred bytes plus the space for its
stack, so the runtime can support literally millions of them, limited
only by the available memory.  Unlike OS threads, the memory used by
Haskell threads is movable, so the garbage collector can pack threads
together tightly in memory and eliminate fragmentation.  Threads can
also expand and shrink on demand, according to the stack demands of
the program.  When using multiple processors, the GHC runtime system
automatically migrates threads between cores in order to balance the
load.

User-space threading is not unique to Haskell, indeed many other
languages, including early Java implementations, have had support for
user-space threads (sometimes called ``green threads'').  It is often
thought that user-space threading hinders interoperability with
foreign code and libraries that are using OS threads, and this is one
reason that OS threads tend to be preferred.  However, with some
careful design it is possible to overcome these difficulties too, as
we shall see in \secref{conc-ffi}.

