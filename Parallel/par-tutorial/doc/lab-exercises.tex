\documentclass[11pt,a4paper]{article}

\usepackage{listings}
\usepackage{color}
\usepackage{code}
\usepackage{graphicx}
\usepackage{hyperref}

\usepackage{natbib}
\bibpunct();A{},
\let\cite=\citep

%include lhs2TeX.fmt

% -----------------------------------------------------------------------------=
% Haskell listing style

% Taken from
%  http://ulissesaraujo.wordpress.com/2008/03/30/latex-listings-package-haskell/
% and modified (quite a lot)

\definecolor{lightgray}{gray}{0.96}
\definecolor{gray_ulisses}{gray}{0.55}
\definecolor{castanho_ulisses}{rgb}{0.71,0.33,0.14}
\definecolor{preto_ulisses}{rgb}{0.41,0.20,0.04}
\definecolor{green_ulises}{rgb}{0.2,0.75,0}

\lstdefinelanguage{HaskellUlisses} {
        basicstyle=\ttfamily\footnotesize,
	sensitive=true,
        morecomment=[l][\color{gray_ulisses}\ttfamily]{--},
        morecomment=[s][\color{gray_ulisses}\ttfamily]{\{-}{-\}},
	morestring=[b]",
	stringstyle=\color{red},
	showstringspaces=false,
        numberblanklines=false,
	showspaces=false,
	breaklines=true,
	showtabs=false,
        backgroundcolor=\color{lightgray},
        emph=
        {[1]
		case,class,data,deriving,do,else,if,import,in,infixl,infixr,instance,let,
                module,of,primitive,then,type,where,foreign,import,export,ccall
        },
        emphstyle={[1]\color{blue}},
	emph=
	{[2]
        },
	emphstyle={[2]\color{castanho_ulisses}},
	emph=
	{[3]
        },
	emphstyle={[3]\color{preto_ulisses}\textbf},
	emph=
	{[4]
        },
	emphstyle={[4]\color{castanho_ulisses}\textbf},
	emph=
	{[5]
        },
	emphstyle={[5]\color{preto_ulisses}\textbf}
}

\lstdefinestyle{numbers}
  {numbers=left, stepnumber=1, numberstyle=\color{gray_ulisses}\tiny, numbersep=10pt}
\lstdefinestyle{nonumbers}
  {numbers=none}

\lstnewenvironment{haskell}
{\lstset{language=HaskellUlisses,style=nonumbers}}
{}

\lstnewenvironment{numhaskell}
{\lstset{language=HaskellUlisses,style=numbers}}
{}

\lstnewenvironment{floathaskell}
{\lstset{language=HaskellUlisses,style=numbers}}
{}



\newcommand{\comment}[1]{}

\newcommand{\Section}[2]{\section{#2}\label{sec:#1}}
\newcommand{\Subsection}[2]{\subsection{#2}\label{sec:#1}}
\newcommand{\Subsubsection}[2]{\subsubsection{#2}\label{sec:#1}}
\newcommand{\secref}[1]{Section~\ref{sec:#1}}
\newcommand{\figref}[1]{Figure~\ref{fig:#1}}
\newcommand{\lstref}[1]{Listing~\ref{lst:#1}}

\title{Parallel and Concurrent Programming in Haskell
  \\ \normalsize{Lab Exercises}}

\author{Simon Marlow\\\texttt{simonmar@microsoft.com}\\Microsoft Research Ltd., Cambridge, U.K.}

\begin{document}

\maketitle
\makeatactive

% \begin{abstract}
% \end{abstract}

\Section{par}{Lab 1: Parallel Haskell}

Lines beginning with @$@ are commands that you type at the
command-line (a terminal in Linux, or a command-prompt in Windows).
For example:

{\small \begin{verbatim}
  $ ghc-pkg list
\end{verbatim}}

\noindent to see what Haskell packages are installed.

First, get the sample code, if you don't already have it.  It can be
downloaded from

{\small \begin{verbatim}
  http://community.haskell.org/~simonmar/par-tutorial.tar.gz
\end{verbatim}}

\noindent or alternatively, get it using @git@:

{\small \begin{verbatim}
  $  git clone git://github.com/simonmar/par-tutorial.git
\end{verbatim}}

\noindent Build the code.  There is a @Makefile@, so if your system
has @make@ installed you should be able to say @make@ to build all the
programs:

{\small \begin{verbatim}
  $ cd par-tutorial/code
  $ make
\end{verbatim}}

\noindent If not, you can compile any individual program like this:

{\small \begin{verbatim}
  $ ghc -threaded -rtsopts -eventlog --make -O2 sudoku1.hs
\end{verbatim}}

\noindent Test that you can get parallel speedup. First, the sequential version:

{\small \begin{verbatim}
  $ ./sudoku1 sudoku17.1000.txt +RTS -s
\end{verbatim}}

\noindent Next, the static parallel version:

{\small \begin{verbatim}
  $ ./sudoku2 sudoku17.1000.txt +RTS -s -N2
\end{verbatim}}

\noindent Next, the @parMap@ version:

{\small \begin{verbatim}
  $ ./sudoku3 sudoku17.1000.txt +RTS -s -N2
\end{verbatim}}

\noindent (the last one should be the fastest)

Now try a larger example, a 16000-problem dataset:

{\small \begin{verbatim}
  $ ./sudoku1 sudoku17.16000.txt +RTS -s
  $ ./sudoku3 sudoku17.16000.txt +RTS -s -N2
\end{verbatim}}

What speedup did you get?  Can you explain the discrepancy? (hint:
look at the @SPARKS@ line in the @+RTS -s@ output.  GHC's spark pool
can only hold about 8000 sparks, new sparks are discarded when the
pool is full).

This example took a while to run, so let's simulate the problem on the
smaller dataset by reducing the spark pool size to 500:

{\small \begin{verbatim}
  $ ./sudoku3 sudoku17.1000.txt +RTS -s -N2 -e500
\end{verbatim}}

Now, to avoid overflowing the spark pool we want to create fewer
sparks, and so each spark needs to do more work.  Hence we need to
divide the work into fewer, but larger, chunks.

@Control.Parallel.Strategies@ provides the following function:

{\small \begin{verbatim}
     parListChunk :: Int -> Strategy a -> Strategy [a]
\end{verbatim}}

The first argument to parListChunk is the chunk size in list elements,
and the second is the Strategy to apply to the list elements.

Modify @sudoku3.hs@ to use @parListChunk@.

Try your version on the 1000-problem dataset with a spark pool size of 500:

{\small \begin{verbatim}
  $ ./sudoku3 sudoku17.1000.txt +RTS -s -N2 -e500
\end{verbatim}}

\noindent (if you want, try it on the 16000-problem dataset too)

What speedup did you get relative to sudoku1?  Try a few different
chunk sizes.  Does it make any difference?

The Strategies library includes another operation for parallelising
list-based operations:

{\small \begin{verbatim}
    parBuffer :: Int -> Strategy a -> Strategy [a]
\end{verbatim}}

Rather than dividing the list into chunks, parBuffer processes the
list in a stream-like way, sparking N elements of the list ahead of
the current element.  At any given point in time there should be no
more than N active sparks.

Convert @sudoku3.hs@ to use @parBuffer@, and try it on the
1000-problem dataset.  What speedup did you get, and how does that
compare to the results with @parListChunk@?

Try a few different buffer sizes - what difference do they make?

\subsection{Experiments with K-Means}

The code for K-Means is in @par-tutorial/code/kmeans@.  Compile it and
run it to make sure you can get some parallel speedup.

Run the program like this:

{\small \begin{verbatim}
 $ ./kmeans seq
\end{verbatim}}

or

{\small \begin{verbatim}
 $ ./kmeans par 1000 +RTS -N2
\end{verbatim}}

to run the parallel version with 1000 chunks on 2 processors.

Note!  This program has to read the data set from disk each time it is
run, which takes a substantial amount of time.  Therefore it includes
code internally to measure the time taken to run the actual algorithm,
and it prints the result at the end.  When computing parallel speedup,
use this time, not the output from +RTS -s.

As written, the program is structured as map/reduce with a single
level.  Instead we could structure it as divide and conquer to any
depth, using the provided functions

\begin{haskell}
  step   :: Int -> [Cluster] -> [Vector] -> [Cluster]
  reduce :: Int -> [[Cluster]] -> [Cluster]
\end{haskell}

\noindent \textbf{Exercise.} Restructure the program to use binary divide-and-conquer.

Hints: the input data set is a list of points and the algorithm
needs to run repeatedly over it, so it will help to build the tree
over which we run the divide-and-conquer algorithm up front.

Use a simple binary tree data structure:

\begin{haskell}
data Tree a = Leaf a
            | Node (Tree a) (Tree a)
\end{haskell}

Write a function to convert the list of points into a tree, for
example:

\begin{haskell}
mkPointTree
   :: Int            -- the depth at which to stop dividing
   -> [Vector]       -- the set of points
   -> Int            -- number of points
   -> Tree [Vector]
\end{haskell}

Next, write a function to implement the divide and conquer:

\begin{haskell}
divconq :: Tree [Vector] -> [Cluster]
\end{haskell}

It will help to make this a local function inside the loop function in
@kmeans_par@, because it needs access to @clusters@, the current
clusters, and @nclusters@.  At each node of the tree, compute the two
branches in parallel.  Use either Strategies or the Eval monad
directly.

Call divconq each time around the loop to compute the new set of
clusters.

For the depth, try 10 (this will give $2^10$, i.e. 1024 leaves).

Run the program - how much speedup do you see?  How does it compare
with the original one-level map/reduce?

Sample answer: @par-tutorial/kmeans/kmeans2.hs@

Bonus: modify the divide-and-conquer version to use the Par monad,

Hints:
\begin{itemize}
\item replace @import Control.Parallel.Strategies@ with
    @import Control.Monad.Par@

\item the @divconq@ function needs to change:

  \begin{haskell}
      divconq :: Tree [Vector] -> Par [Cluster]
  \end{haskell}

    \noindent and use runPar when calling divconq.  For defining @divconq@ you
    can use these from the @Control.Monad.Par@ library:

  \begin{haskell}
      spawn :: a -> Par (IVar a)
      get   :: IVar a -> Par a
  \end{haskell}
\end{itemize}

Sample answer: @par-tutorial/kmeans/kmeans3.hs@

\newpage\Section{conc}{Lab 2: Concurrent Haskell}

The sample code corresponding to the examples in the notes can be
found in @par-tutorial/code@.  Try the @fork@ example from
Section~3.1:

{\small \begin{verbatim}
  $ ghc -threaded -rtsopts -eventlog --make fork.hs
  $ ./fork
  ABABABABABABABABABABABABABABABA...
\end{verbatim}}

\textbf{Exercise 1.} In @par-tutorial/code/bingtranslator.hs@ there is
a program that translates a line of text into multiple languages using
the Bing Translate API.  For example:

{\small \begin{verbatim}
$ ./bingtranslator "translate this"
"translate this" appears to be in language "en"
ar: ترجمة هذا
bg: превод на това
zh-CHS: 翻译这
zh-CHT: 翻譯這
cs: přeložit
da: oversætte dette
... etc.
\end{verbatim}}

\noindent Compile the program - for this you may
need to install the @xml@ and @utf8-string@ packages first:

{\small \begin{verbatim}
  $ cabal install xml utf8-string
  $ ghc -threaded -rtsopts -eventlog --make bingtranslator.hs
\end{verbatim}}

When the program is compiled, run it yourself with some sample
text\footnote{On Windows you may not see the international characters
  appear correctly on the console.  First, switch the codepage to
  UTF-8 with @chcp 65001@.  Then redirect the output of
  @bingtranslator@ to a file, and open the file in Notepad to see the
  output correctly.} . Note how the translations appear slowly - the program queries the Bing API
for each translation in sequence.

\begin{itemize}
\item Convert the program to perform all the translations
  concurrently.  Use the @Async@ API (the code is in the
  @bingtranslator.hs@ source file).
\item The program also makes two initial queries to the Bing API: one
  to get the list of supported languages, and another to detect the
  language in which the initial text is written.  Make these two
  queries concurrently.
\end{itemize}

A sample solution can be found in @bingtranslatorconc.hs@.

\textbf{Exercise 2.} Implement a bounded channel type, with the
following signatures:

\begin{haskell}
data BoundedChan
newBoundedChan  :: Int -> IO (BoundedChan a)
readBoundedChan :: BoundedChan a -> IO a
writeBoundedChan :: BoundedChan a -> a -> IO ()
\end{haskell}

\noindent @newBoundedChan@ takes the size of the channel, and
@writeBoundedChan@ blocks if the channel is full.

Use either @MVar@ or @STM@, or try both for comparison.

\textbf{Exercise 3.} (for the performance-obsessed) Write a program in which
two threads communicate over a channel, one writing a large number of
items to the channel and the other reading them.  Measure the time it
takes, using (a) the @Chan@ type, (b) the @TChan@ type, (c) the
bounded channel from Exercise 1 (choosing a suitable bound).  Can you
explain the differences in performance?

\textbf{Exercise 4.} Implement the @MVar@ type using STM, in the most
obvious way:

\begin{haskell}
newtype TMVar a = TMVar (TVar (Maybe a))
newEmptyTMVar :: STM (TMVar a)
takeTMVar     :: TMVar a -> STM a
putTMVar      :: TMVar a -> a -> STM ()
\end{haskell}

This @MVar@ variant does not have the properties of \emph{fairness}
and {single-wakeup} that the standard @MVar@ type does.  Can you
modify the STM version so that it provides these properties?

\textbf{Exercise 5.} The operation @wait@ in the @Async@ API is not
async-exception-safe.  Can you make it safe?

%      - make a channel implementation using Data.Seq.  Is it faster?
%      - modify the server to only allow a fixed number of clients
%      - write a program to generate load for the server. How many
%        concurrent connections can it cope with?
%      - geturlscancel: wait isn't async-safe, make it so

\end{document}
